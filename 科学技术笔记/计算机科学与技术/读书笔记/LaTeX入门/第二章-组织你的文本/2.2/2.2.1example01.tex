%-*- encoding: UTF-8 -*-
% example.tex
% 例子
\documentclass[UTF8]{ctexart}

\title{2.2.1 正文段落}
\author{张三}
\date{\today}

\setlength\parskip{0pt}

%全局地设置断点列表
\hyphenation{man-u-script com-pu-ter gym-na-sium}

\begin{document}

\maketitle

\noindent
中文文档类每段缩进,并自动设置段落缩进为两个汉字宽度。如果在某一段开头临时禁用缩进,可以在段前使用noindent命令;而如果要在文本本来没有缩进的地方使用缩进,可以使用
\indent
命令产生一个长为parindent的缩进。

除了段落首行缩进,另一个关于分段的重要参数是段与段之间的垂直距离,这是由变量parskip控制。

段落最明显的属性是对齐方式。LaTeX的段落默认是两端均匀对齐的,也可以改为左对齐、右对齐或居中格式。raggedright命令设置段落左对齐。

\raggedright
English words like `technology' stem from a Greek root beginning with the letters tec\dots

\raggedleft
\date{\today}

\centering
以上三种对齐方式的用法。

\raggedright
LaTeX提供了三个环境来排版不同对齐方式的文字:flushleft环境左对齐、flushright环境右对齐、center环境居中。

\begin{flushleft}
左对齐环境
\end{flushleft}

\begin{flushright}
右对齐环境
\end{flushright}

\begin{center}
居中环境
\end{center}

TeX的断词算法通常工作得很好,不需要人工干预,不过任然可能会有一些特殊的单词是TeX不能正确处理的,此时可以在单词中使用-命令告诉LaTeX可能的断点,如man\-u\-script。还可以在使用hyphenation命令在导言区全局地设置断点列表。

\begin{sloppypar}
使用sloppy命令可以允许段落中更大的空格,从而禁用断词功能;与之对应的命令是fussy,让段落恢复默认的较严格的间距。更多地是使用等效的sloppypar环境,把允许更宽松间距的文本段落放在环境中。
\end{sloppypar}

三个宏包hyphenat、ragged2e、microtype与断词有关。

\setlength\leftskip{4em}
\setlength\rightskip{4em}
These parameters tell \TeX{} how much glue to place at the left and at the right end of each line of the current paragraph.

\end{document}
